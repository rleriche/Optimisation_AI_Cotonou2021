%%%%%%%%%%%%%%%%%%%%%%%%%%%%%%%%%%%%%%%%%%%%%%%%%%%%%%%%%%%%%%%%%%%%%
%
% Enonce de projet, Cotonou Juillet 2021
%
%%%%%%%%%%%%%%%%%%%%%%%%%%%%%%%%%%%%%%%%%%%%%%%%%%%%%%%%%%%%%%%%%%%%%

\documentclass[12pt]{article}
\usepackage[utf8]{inputenc}
\usepackage{amsmath}
\usepackage{amsfonts}
\usepackage{amssymb}
\usepackage{graphicx}
\graphicspath{{figures/}}
%\usepackage{beamerthemesplit}
%\usepackage{beamerthemeshadow} 
\usepackage{color}
\usepackage{hyperref}
\usepackage{xspace}
\usepackage{xifthen}
\usepackage{multicol}
\usepackage{mathtools}
\usepackage{algorithm,algorithmic}
\usepackage{dsfont}
%
% these 2 next needed for mathbb greek letters
\usepackage{breqn} 
\usepackage[bbgreekl]{mathbbol}
%
\usepackage{bbm} % this one is needed for the indicator
%

% custom commands in sty file, for easier writting and change of notations
\usepackage{my_notations}

\title{Enoncé du projet pour le cours d'optimisation}
\date{Juillet 2021 \\
Ecole d'Eté en Intelligence Artificielle \\
fondation Vallet\\
Cotonou, Bénin}
\author{}

%%%%%%%%%%%%%%%%%%%%%%%%%%%%%%%%%%%%%%%%%%%%%%%%%%%%%%%%%%%%%%%%%%%%%
\begin{document}
\maketitle

\section{Prise en main du code}
Aller dans le répertoire \texttt{Project}. 
\begin{enumerate}
\item Ouvrir \texttt{3Dplots}. C'est un fichier pour dessiner des fonctions en \texttt{d<-2} dimensions (``contour plots'' et ``plot 3D''). 
Dessiner plusieurs des fonctions données dans \texttt{test\_functions}\footnote{Les fonctions de \texttt{test\_functions} marchent avec des dimensions arbitraires}.
Il suffit de changer le champ \texttt{fun<-} et mettre le nom de la fonction à dessiner
(par exemple \texttt{fun<-quadratic} ou \texttt{fun<-rosen} ou \texttt{fun<-ackley}, \ldots). \\
Repérer quelles fonctions sont multimodales (i.e., ont plusieurs optima locaux différents).
\item Ouvrir \texttt{mainOptim}. Il s'agit du programme principal qui permet de 
\end{enumerate}
\section{Création d'une nouvelle fonction}
\section{Ajout de restart à une méthode de descente}

\end{document}
